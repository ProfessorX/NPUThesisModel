\begin{cnabstract}
计算机和数字通信已经对当今社会生活的各个方面产生了前所未有的影响。为了满足人们对数字通信服务日益增长的需求,
工程师必须采取便捷可行、经济适用的系统分析设计方法。在数字通信领域,从传输内容到传输媒介以及各种辅助软件,
我们能看到这些可用技术数量上的大规模增长。但是在炙手可热的系统性能设计、分析和优化领域,经典好用的技术并不多,
实际工程效果也不尽如人意。在过去的几十年里,形形色色的计算机辅助工程技术被开发出来用于设计复杂系统。这些计算机辅助
工程技术十分依赖于对复杂系统各个部件及系统本身的建模、分析和仿真---这是持续整个项目生命周期的过程。在电子计算机
走进千家万户的今天,对通信系统领域的计算机辅助设计、分析和模拟进行更加深入和细致的研究,实不为过。
\par
本文通过对数字通信系统领域一些基本概念的探讨、基本方法的回顾和几个基础模拟方法的比较,研究了该领域的几个重要方向。
本文的核心部分是蒙特--卡罗方法、仿射投影算法和其他一些方法在通信系统的中间环节---信道部分的分析。并且,为了更好地
将比较结果展示出来,本文列举出了几种图形化的分析结果---从本文作者毕业设计的研究经历来看,在通信系统领域图形化是对
研究成果一种极好的表示。
\par
% 尽管作者在指导老师的项目初期就加入了他的科研团队,参与国家自然科学基金资助的自适应滤波器课题研究,由于设计时间实在
% 有限,本文的可参考价值没有达到最开始的目标。

\cnkeywords{仿射投影算法,性能分析,信道评估,通信,MATLAB}
\end{cnabstract}

\begin{enabstract}
\noindent Digital communications and computers are having a tremendous impact on the world today. In order to meet the increasing demand for digital communication services, engineers must design systems in a timely and cost-effective manner. The number of technologies available for providing a given service is growing daily, covering transmission media, devices, and software. The resulting design, analysis and optimization of performance can be very demanding and difficult. Over the past decades, a large body of computer-aided engineering techniques have been developed to facilitate the design process of complex technological systems. These techniques rely on models of devices and systems, both analytic and simulation, to guide the analysis and design throughout the life cycle of a system. Computer-aided design, analysis and simulation of communication systems constitute a new and important part of this process.
\par
This thesis studies different aspects of the simulation of communication systems by covering some basic ideas, approaches and methodologies within the simulation context. Performance measurement of a digital communication is the main focus of this thesis. However, some popular visual indicators of signal quality, which are often generated in a simulation to provide a qualitative sense of the performance of a digital system, are also considered.
\par
Another purpose of this thesis is to serve as a model for developing simulation or template of other systems. In other words, peer researchers learning to simulate a system can use the work presented here as a starting point.
\par
\enkeywords{affine projection algorithm, performance evaluation, channel estimation, communication, MATLAB}
\end{enabstract} 