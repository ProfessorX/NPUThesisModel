\chapter{源代码}
部分源代码重合程度较高,在此只列出核心的几个程序作为范例。
\section{AffineProjection.m}
\begin{lstlisting}
function    [outputVector,...
             errorVector,...
             coefficientVector] =   Affine_projection(desired,input,S)

%   Affine_projection.m
%   Implements the Complex Affine-Projection algorithm for COMPLEX valued data.
%
%   Syntax:
%       [outputVector,errorVector,coefficientVector] = Affine_projection(desired,input,S)
%
%   Input Arguments:
%       . desired   : Desired signal.                               (ROW vector)
%       . input     : Signal fed into the adaptive filter.          (ROW vector)
%       . S         : Structure with the following fields
%           - step                  : Convergence (relaxation) factor.
%           - filterOrderNo         : Order of the FIR filter.
%           - initialCoefficients   : Initial filter coefficients.  (COLUMN vector)
%           - gamma                 : Regularization factor.
%                                     (small positive constant to avoid singularity)
%           - memoryLength          : Reuse data factor.
%                                     (referred as L in the textbook)
%
%   Output Arguments:
%       . outputVector      :   Store the estimated output of each iteration.   (COLUMN vector)
%       . errorVector       :   Store the error for each iteration.             (COLUMN vector)
%       . coefficientVector :   Store the estimated coefficients for each iteration.
%                               (Coefficients at one iteration are COLUMN vector)
%

%   Some Variables and Definitions:
%       . prefixedInput         :   Input is prefixed by nCoefficients -1 zeros.
%                                   (The prefix led to a more regular source code)
%
%       . regressor             :   Auxiliar variable. Store the piece of the
%                                   prefixedInput that will be multiplied by the
%                                   current set of coefficients.
%                                   (regressor is a COLUMN vector)
%
%       . nCoefficients         :   FIR filter number of coefficients.
%
%       . nIterations           :   Number of iterations.


%   Initialization Procedure
nCoefficients       =   S.filterOrderNo+1;
nIterations         =   length(desired);

%   Pre Allocations
errorVectorApConj       =   zeros((S.memoryLength+1)
,nIterations);
outputVectorApConj      =   zeros((S.memoryLength+1)
,nIterations);
coefficientVector       =   zeros(nCoefficients
,(nIterations+1));
regressor               =   zeros(nCoefficients
,(S.memoryLength+1));

%   Initial State Weight Vector
coefficientVector(:,1)  =   S.initialCoefficients;

%   Improve source code regularity
prefixedInput           =   [zeros((nCoefficients-1),1)
                             transpose(input)];
prefixedDesired         =   [zeros(S.memoryLength,1)
                             transpose(desired)];

%   Body
for it = 1:nIterations,

    regressor(:,2:S.memoryLength+1) =   regressor(:,1:S.memoryLength);

    regressor(:,1)                  =
    prefixedInput(it+(nCoefficients-1):-1:it);

    outputVectorApConj(:,it)        =
    (regressor')*coefficientVector(:,it);

    errorVectorApConj(:,it)         =
    conj(prefixedDesired(it+(S.memoryLength):-1:it))...
    -outputVectorApConj(:,it);

    coefficientVector(:,it+1)       =
    coefficientVector(:,it)+(S.step*regressor*...
    inv(regressor'*regressor+S.gamma*...
    eye(S.memoryLength+1))*errorVectorApConj(:,it));

end

outputVector            =   outputVectorApConj(1,:)';
errorVector             =   errorVectorApConj(1,:)';

%   EOF
\end{lstlisting}

\section{AffineProjectionExample.m}
\begin{lstlisting}
%%%%%%%%%%%%%%%%%%%%%%%%%%%%%%%%%%%
%                        Example: System Identification                         %
%%%%%%%%%%%%%%%%%%%%%%%%%%%%%%%%%%%
%                                                                               %
%  In this example we have a typical system identification scenario. We want    %
% to estimate the filter coefficients of an unknown system given by Wo. In      %
% order to accomplish this task we use an adaptive filter with the same         %
% number of coefficients, N, as the unkown system. The procedure is:            %
% 1)  Excitate both filters (the unknown and the adaptive) with the signal      %
%   x. In this case, x is chosen according to the 4-QAM constellation.          %
%   The variance of x is normalized to 1.                                       %
% 2)  Generate the desired signal, d = Wo' x + n, which is the output of the    %
%   unknown system considering some disturbance (noise) in the model. The       %
%   noise power is given by sigma_n2.                                           %
% 3)  Choose an adaptive filtering algorithm to govern the rules of coefficient %
%   updating.                                                                   %
%                                                                               %
%     Adaptive Algorithm used here: Affine Projection                           %
%                                                                               %
%%%%%%%%%%%%%%%%%%%%%%%%%%%%%%%%%%%


%   Definitions:
ensemble    = 100;                          % number of realizations within the ensemble
K           = 500;                          % number of iterations
H           = [0.32+0.21*j,-0.3+0.7*j,0.5-0.8*j,0.2+0.5*j].';
Wo          = H;                            % unknown system
sigma_n2    = 0.04;                         % noise power
N           = 4;                            % number of coefficients of the adaptive filter
mu          = .1;                           % convergence factor (step)  (0 < mu < 1)
gamma       = 1e-12;                        % small positive constant to avoid singularity
L           = 1;                            % data reuse factor (L=0 -> NLMS; L=1 -> BNLMS; ...)


%   Initializing & Allocating memory:
W       = ones(N,(K+1),ensemble);   % coefficient vector for each iteration and realization; w(0) = [1 1 1 1].'
MSE     = zeros(K,ensemble);        % MSE for each realization
MSEmin  = zeros(K,ensemble);        % MSE_min for each realization


%   Computing:
for l=1:ensemble,

    X       = zeros(N,1);               % input at a certain iteration (tapped delay line)
    d       = zeros(1,K);               % desired signal

    x        = (sign(randn(K,1)) + j*sign(randn(K,1)))./sqrt(2); % Creating the input signal (normalized)
    sigma_x2 = var(x);                                           % signal power = 1
    n        = sqrt(sigma_n2/2)*(randn(K,1)+j*randn(K,1));       % complex noise

    for k=1:K,

        X       =   [x(k,1)
                     X(1:(N-1),1)];              % input signal (tapped delay line)

        d(k)    =   (Wo'*X(:,1))+n(k);           % desired signal

    end

    S   =   struct('step',mu,'filterOrderNo',(N-1),'initialCoefficients',...
                   W(:,1,l),'gamma',gamma,'memoryLength',L);
    [y,e,W(:,:,l)]  =   Affine_projection(d,transpose(x),S);

    MSE(:,l)    =   MSE(:,l)+(abs(e(:,1))).^2;
    MSEmin(:,l) =   MSEmin(:,l)+(abs(n(:))).^2;

end


%   Averaging:
W_av = sum(W,3)/ensemble;
MSE_av = sum(MSE,2)/ensemble;
MSEmin_av = sum(MSEmin,2)/ensemble;


%   Plotting:
figure,
plot(1:K,10*log10(MSE_av),'-k');
title('Learning Curve for MSE');
xlabel('Number of iterations, k'); ylabel('MSE [dB]');

figure,
plot(1:K,10*log10(MSEmin_av),'-k');
title('Learning Curve for MSEmin');
xlabel('Number of iterations, k'); ylabel('MSEmin [dB]');

figure,
subplot 211, plot(real(W_av(1,:))),...
title('Evolution of the 1st coefficient (real part)');
xlabel('Number of iterations, k'); ylabel('Coefficient');
subplot 212, plot(imag(W_av(1,:))),...
title('Evolution of the 1st coefficient (imaginary part)');
xlabel('Number of iterations, k'); ylabel('Coefficient');
\end{lstlisting}

\section{AffineProjectionLearningCurve.m}
\begin{lstlisting}

%%% APA Algorithm for System Identification


% % %% number of runs, iterations...
% % num_runs=200;  num_iter=500;
% %
% %
% % %% Input type
% % GAUSSIAN_WHITE   = 1;
% % GAUSSIAN_COLORED = 2;
% % UNIFORM_WHITE    = 3;
% % UNIFORM_COLORED  = 4;
% % input_type = GAUSSIAN_WHITE; %UNIFORM_WHITE;
% %
% %
% % %% APA parameters
% % M = 16;    % input vector length
% % K = 4;     % number of past regressors, observ. to use
% % D = 8;     % delay between past regressors, observ. used
% % mu = 1.0;  % update gain parameter


%% Plant parameters
%v_var = 0.001;             % measurement noise (for 30dB)
epsil_I = 0;%0.001*eye(K);    % regularization parameter

% randomly generate optimum weights
w_opt = rand(M, 1);

%% For input signal statistics...
E_r2 = 0;
R = zeros(M,M);

%% For numerous runs of plant...
START_DLY = M + K*D;
wgt_vec = zeros(M,1);
mse_vec = zeros(num_iter,1);
for run_cnt = 1:num_runs

    %% Generate plant signal (before noise)
    if input_type == GAUSSIAN_WHITE
        u_i = randn(1, num_iter+START_DLY);         % gaussian, unit variance
    elseif input_type == UNIFORM_WHITE
        u_i = 2*rand(1, num_iter+START_DLY,1) - 1;  % uniform, between -1, +1
    elseif input_type == GAUSSIAN_COLORED
        u_i = randn(1, num_iter+START_DLY);         % gaussian, unit variance
        u_i = filter(1, [1 -0.9], u_i);             % pole at 0.9
    else % UNIFORM_COLORED
        u_i = 2*rand(1, num_iter+START_DLY,1) - 1;  % uniform, between -1, +1
        u_i = filter(1, [1 -0.9], u_i);             % pole at 0.9
%       u_i = filter(1, [1 -0.5], u_i);             % pole at 0.5
    end
    y_i = filter(w_opt, 1, u_i);                    % noiseless plant output


    % Find autocorr. R, E[1/r^2], for input signal, u_i
    for i = START_DLY : (num_iter+START_DLY-1)
        u_vec = u_i(i:-1:i-M+1).';
        R = R + (u_vec * u_vec') / num_iter;
        E_r2 = E_r2 + (1/(norm(u_vec)^2)) / num_iter;
    end


    %% Add measurement noise to plant output
    scale_y = 1 / sqrt(mean(y_i.*conj(y_i)));
    y_i = y_i * scale_y;                             % normalize plant power (x4)
    v_i = sqrt(v_var)*randn(1, num_iter+START_DLY);  % noise for 30dB SNR
    d_i = y_i + v_i;                                 % add noise to output

    %% Initialize APA
    w = zeros(M, 1);           % init. weights
    U_i = zeros(K, M);
    D_i = zeros(K, 1);

    %% APA algorithm computation
    for i = START_DLY : (num_iter+START_DLY-1)
        % APA inputs (U_i and D_i)
        for k = 0 : (K-1);
            i_a = i - k*D;
            i_b = i - k*D - M+1;
            U_i(k+1,:) = u_i(i_a : -1 : i_b);
            D_i(k+1,:) = d_i(i_a);
        end

        % Apply APA algorithm
        Y_apa = U_i*w;
        e_i = D_i - Y_apa;
        w_p = w;
        w = w + mu*U_i'*inv(epsil_I + U_i*U_i')*e_i;

        % Store MSE output
        mse_vec(i-START_DLY+1) = mse_vec(i-START_DLY+1) + e_i(1).*conj(e_i(1));

    end

    %% Add final weights to average
    wgt_vec = wgt_vec + w;
end


%% Average MSE error, final weight  vector
mse_vec = mse_vec / num_runs;
wgt_vec = wgt_vec / num_runs;

% various parameters (R, E[1/r^2], etc...)
E_r2 = E_r2 / num_runs;
R = R / num_runs;
tr_R = trace(R);
[U,S,V] = svd(R);
p_v = diag(S)/tr_R;
%  p_v = ones(size(p_v))*(1/M);

% % %% Plot learning curves
% % figure; plot(0:num_iter-1, 10*log10(mse_vec));
% % title('APA MSE (Ensemble-averaged, 20 Runs, 1000 Iter.)');
% % xlabel('Iteration, n'); ylabel('Mean-Square-Error, J(n)');
% % axis([-10 510 -30 0]); grid on;


%%%%%%%%%%%%%%%%%%%%%%%%%%%%%%
format long;

% Display simulation MSE results
K__D__mu =[K D mu]
msevec_sim = mse_vec(end-3:end).'                                   ;
msevec_avg = mean(mse_vec(end-floor(num_iter*0.2):end))             ;
E_r2

% Compute component for correl. of noise, coeff.
G = 0;
for i = 1:M
    for g = 1 : (K-1)
        G = G + p_v(i) * ( 1 - (1-(1-p_v(i))^g)/(1-(1-p_v(i))^K) ) * (1-mu)^g;
    end
end

% Compute theoretical MSE
mse_theory_a = v_var + (mu*v_var)/(2-mu)*E_r2*tr_R                              ;
mse_theory_b = v_var + (mu*v_var)/(2-mu)*E_r2*tr_R*(1+2*G)                      ;
mse_theory_diniz = v_var + K*(mu*v_var)/(2-mu)*(1-(1-mu)^2)/(1-(1-mu)^(2*K))    ;


%%%%%%%%%%%%%%%%%%%%%%%%


% Theoretical MSE curve
a = mu*(2-mu);
lmb_ni_orig = zeros(M,num_iter);
lmb_ni_mod  = zeros(M,num_iter);
lmb_ni_orig(:,1) = diag(V'*w_opt*w_opt'*V)*scale_y^2;  % transformed (based on Rxx)
lmb_ni_mod(:,1)  = diag(V'*w_opt*w_opt'*V)*scale_y^2;  % transformed (based on Rxx)
mse_vec_theory_orig = zeros(num_iter,1);
mse_vec_theory_mod  = zeros(num_iter,1);

for n = 2 : num_iter
    % find transformed covariance matrix
    for i = 1:M

        % beta term
        b = 1-(1-p_v(i))^K;

        % original formula...
        lmb_ni_orig(i,n) = (1 - a*b)*lmb_ni_orig(i,(n-1)) + mu^2*v_var*E_r2*b;

        % modified formula...
        G_i = 0;
        for g = 1 : (K-1)
            G_i = G_i + ( (1-p_v(i))^g - (1-p_v(i))^K ) * (1-mu)^g;
        end
        lmb_ni_mod(i,n)  = (1 - a*b)*lmb_ni_mod(i,(n-1)) + mu^2*v_var*E_r2*(b+2*G_i);
    end
    % compute theoretical MSE curves
    mse_vec_theory_orig(n) = v_var + tr_R*sum(p_v.*lmb_ni_orig(:,n));     % assumes Rxx = I
    mse_vec_theory_mod(n)  = v_var + tr_R*sum(p_v.*lmb_ni_mod(:,n));      % assumes Rxx = I
end

% Display results
msevec_theory_orig = mse_vec_theory_orig(end-3:end).'               ;
msevec_theory_mod  = mse_vec_theory_mod(end-3:end).'                ;

%pause

format short;

\end{lstlisting}



\section{AffineProjectionSteadyState.m}
\begin{lstlisting}

%%% APA Algorithm for System Identification


% % %% number of runs, iterations...
% % num_runs=200;  num_iter=500;
% %
% %
% % %% Input type
% % GAUSSIAN_WHITE   = 1;
% % GAUSSIAN_COLORED = 2;
% % UNIFORM_WHITE    = 3;
% % UNIFORM_COLORED  = 4;
% % input_type = GAUSSIAN_WHITE; %UNIFORM_WHITE;
% %
% %
% % %% APA parameters
% % M = 16;    % input vector length
% % K = 4;     % number of past regressors, observ. to use
% % D = 8;     % delay between past regressors, observ. used
% % mu = 1.0;  % update gain parameter


%% Plant parameters
%v_var = 0.001;             % measurement noise (for 30dB)
epsil_I = 0;%0.001*eye(K);    % regularization parameter

% randomly generate optimum weights
w_opt = rand(M, 1);

%% For input signal statistics...
E_r2 = 0;
R = zeros(M,M);

%% For numerous runs of plant...
START_DLY = M + K*D;
wgt_vec = zeros(M,1);
mse_vec = zeros(num_iter,1);
for run_cnt = 1:num_runs

    %% Generate plant signal (before noise)
    if input_type == GAUSSIAN_WHITE
        u_i = randn(1, num_iter+START_DLY);         % gaussian, unit variance
    elseif input_type == UNIFORM_WHITE
        u_i = 2*rand(1, num_iter+START_DLY,1) - 1;  % uniform, between -1, +1
    elseif input_type == GAUSSIAN_COLORED
        u_i = randn(1, num_iter+START_DLY);         % gaussian, unit variance
%       u_i = filter(1, [1 -0.9], u_i);             % pole at 0.9
        u_i = filter(1, [1 -0.5], u_i);             % pole at 0.5
    else % UNIFORM_COLORED
        u_i = 2*rand(1, num_iter+START_DLY,1) - 1;  % uniform, between -1, +1
        u_i = filter(1, [1 -0.9], u_i);             % pole at 0.9
%       u_i = filter(1, [1 -0.5], u_i);             % pole at 0.5
    end
    y_i = filter(w_opt, 1, u_i);                    % noiseless plant output


    % Find autocorr. R, E[1/r^2], for input signal, u_i
    for i = START_DLY : (num_iter+START_DLY-1)
        u_vec = u_i(i:-1:i-M+1).';
        R = R + (u_vec * u_vec') / num_iter;
        E_r2 = E_r2 + (1/(norm(u_vec)^2)) / num_iter;
    end


    %% Add measurement noise to plant output
    scale_y = 1 / sqrt(mean(y_i.*conj(y_i)));
    y_i = y_i * scale_y;                             % normalize plant power (x4)
    v_i = sqrt(v_var)*randn(1, num_iter+START_DLY);  % noise for 30dB SNR
    d_i = y_i + v_i;                                 % add noise to output

    %% Initialize APA
    w = zeros(M, 1);           % init. weights
    U_i = zeros(K, M);
    D_i = zeros(K, 1);

    %% APA algorithm computation
    for i = START_DLY : (num_iter+START_DLY-1)
        % APA inputs (U_i and D_i)
        for k = 0 : (K-1);
            i_a = i - k*D;
            i_b = i - k*D - M+1;
            U_i(k+1,:) = u_i(i_a : -1 : i_b);
            D_i(k+1,:) = d_i(i_a);
        end

        % Apply APA algorithm
        Y_apa = U_i*w;
        e_i = D_i - Y_apa;
        w_p = w;
        w = w + mu*U_i'*inv(epsil_I + U_i*U_i')*e_i;

        % Store MSE output
        mse_vec(i-START_DLY+1) = mse_vec(i-START_DLY+1) + e_i(1).*conj(e_i(1));

    end

    %% Add final weights to average
    wgt_vec = wgt_vec + w;
end


%% Average MSE error, final weight  vector
mse_vec = mse_vec / num_runs;
wgt_vec = wgt_vec / num_runs;

% various parameters (R, E[1/r^2], etc...)
E_r2 = E_r2 / num_runs;
R = R / num_runs;
tr_R = trace(R);
[U,S,V] = svd(R);
p_v = diag(S)/tr_R;
%  p_v = ones(size(p_v))*(1/M);

% % %% Plot learning curves
% % figure; plot(0:num_iter-1, 10*log10(mse_vec));
% % title('APA MSE (Ensemble-averaged, 20 Runs, 1000 Iter.)');
% % xlabel('Iteration, n'); ylabel('Mean-Square-Error, J(n)');
% % axis([-10 510 -30 0]); grid on;


%%%%%%%%%%%%%%%%%%%%%%%%%%%
format long;

% Display simulation MSE results
K__D__mu =[K D mu]
msevec_sim = mse_vec(end-3:end).'		;
msevec_avg = mean(mse_vec(end-999:end))		;
E_r2
%p_v
R

% Compute component for correl. of noise, coeff.
G = 0;
for i = 1:M
    for g = 1 : (K-1)
        G = G + p_v(i) * (1-(1-p_v(i))^g)/(1-(1-p_v(i))^K) * (1-p_v(i))^(K-g) * (1-mu)^(K-g);
%%        G = G + p_v(i) * (1-(1-p_v(i))^g)/(1-(1-p_v(i))^K) * (1-mu)^(K-g);
    end
end

% Compute theoretical MSE
mse_theory_orig_8 = v_var + (mu*v_var)/(2-mu)*E_r2*tr_R				;
mse_theory_mod_23  = v_var + (mu*v_var)/(2-mu)*E_r2*tr_R*(1+2*G)		;
mse_theory_shin_23 = v_var + K*(mu*v_var)/(2-mu)*E_r2*tr_R			;
mse_theory_diniz = v_var + K*(mu*v_var)/(2-mu)*(1-(1-mu)^2)/(1-(1-mu)^(2*K))	;

format short;

\end{lstlisting}

